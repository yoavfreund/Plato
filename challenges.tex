



\reminder{raw form}

\subsection{Research Challenges}
\begin{itemize}
%
\item Specify declarative query languages that process models.
%
\item The database algorithms can work efficiently directly on the models. Specify a logically/physically-separated architecture where an optimizer is aware of the specifics of the model representation and chooses query processing algorithms accordingly.  For example, consider two models represented by their Fourier transform and a query that asks for their correlation. It is most efficient to compute directly on the frequency domain rather than bringing back to time domain.
%
\item Signal processing community provides a wide variety of model generators, with various guarantees on the loss information. Plato must semiautomate the choice of the appropriate model generator. This requires a classification of model generators with respect to their loss of information properties. More importantly, different queries may pose different needs of accuracy and of what constitues information loss. The semiautomation must take into account query workload. The query answering must utilize the best model for the problem. Granularity of queries can also play a role: Can you avoid computing the entire model and instead compute on the fly only the parts of the model that are of interest to the query? Which models are best fo such?
%
\item How do you incrementally maintain the model as the data change? Tradeoff between speed of convergence and computational resources used. Again, query workload must specify.
\end{itemize}

