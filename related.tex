\section{Related Work}

\paragraph{Priori work from signal processing and machine learning}
Work on adaptive signal processing starts in the 50's with the
the Wiener filter, followed by the Kalman filter and the Widrow-Hoff
algorithm~\cite{adaptive_signal_processing}. Enormous progress has
been made since that time and adaptive signal processing is now
textbook material~\cite{dsp,adaptive_filter_theory}. Closely related
is data compression theory and practice~\cite{DBLP:books/mk/Sayood12}.
Recent work of the database field in this area is summarized in \cite{DBS-005}.

The last decade has seen the rapid development in the area of online
learning. This is a subarea of machine learning that is based on ideas
from adaptive signal processing and has made interesting and fruitful
connection to information theory and game theory. PI-Freund is active
in this area of research~\cite{prediction_learning_models}.

\paragraph{Model-based data infrastructure}
The idea of incorporating models in database systems was first presented in the context of MauveDB \cite{mauvedb-grid, mauvedb-cidr, mauvedb-vldb}. This proposal extends these ideas in several important directions: First, MauveDB argues that models need to be discretized in the coordinates' grid, before they can be queried. In this proposal we show that a fully virtual approach, where the model is perceived as a continuous function, is both easier for the data analyst and more opportune for the query optimizer. For instance, consider two models represented by their Fourier transform and a query that asks for their correlation. It is most efficient to compute this query directly on the frequency domain rather than bring it back to the time domain. Second, while MauveDB showcases some of the challenges that arise in model-based systems and presents solutions for specific models, it lacks a general API that would allow one to plug in arbitrary models (which is one of the main goals of Plato). 
\reminder{To YP: Are we still claiming the following: Did not investigate the connections between (a) choosing models and query workload and (b) query guarantees given chosen models?}

Apart from MauveDB, several works studied particular point problems related to the use of models to represent sensor data, such as comparing existing models in terms of compression or designing indices that would be useful for such models \cite{aberer-cloud, aberer-compression}. However, neither of these works described a general extensible database platform that can accomodate and offer extensive query capabilities over a variety of models.

\paragraph{Efficient query processing on models} The idea of evaluating queries directly on the representation of a model without discretizing them first, was presented in the context of FunctionDB \cite{functiondb}. The work showed that for a broad class of polynomial functions, faster processing is achieved by evaluating queries directly on the algebraic representation of the functions. Plato's goal is to provide the platform infrastructure that enables such optimizations for additional classes of statistical models.

