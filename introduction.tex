
\section{Introduction}
\label{sec:introduction}
Since the 80s, SQL database systems gradually became the foundation of Business Intelligence (BI) and On Line Analytical Processing (OLAP) applications. Their success is primarily attributed to {\em declarative querying} (via SQL), {\em automatic query optimization} and {\em physical-logical independence}, which are explained in Section~\ref{sec:healthy-aspects}. These healthy features of SQL led to increased developer and analyst productivity. They also led to specialized roles for the participants, whereas each participant needs to master only one layer of the overall architecture. 

However, SQL systems fail in the space of predictive analytics that involve sensors generating measurements in the space and time dimensions.%
\footnote{Indeed, one may argue that some of the discussed failures and respective remedies proposed by Plato, also apply to predictive analytics of non-sensor, non-spatiotemporal data. Nevertheless, the focus of this project is exclusively on spatiotemporal sensor data.
}
Section~\ref{sec:the-failure} argues that at the core of the failure of SQL databases in the management and analytics of sensor spatiotemporal data is the lack of a critical abstraction, which is the {\em Real World Model}.%
\footnote{The real world model is frequently called the ``physical model" as it corresponds to the physics of the real world. This proposal uses the term ``real world model" (instead of the term ``physical model") to avoid the confusion caused by the two different meanings of the word ``physical" in the sensors' signal processing and the database community.
}
The signal processing area has been utilizing the real world model concept, with great results in the cleaning, compression and reconstruction of signals. However, this area's approaches have been applicable only to problems with low diversity and low volume of data. 
% Furthermore they require highly knowledgeable signal processing experts, who must combine statistics.
Currently there is a split between database systems and signal processing, which leads to low productivity practices that make ad-hoc combinations of databases with signal processing and statistics. Such practices do not scale in scenarios where many diverse types of data must be quickly analyzed. 

<<<<<<< HEAD
Section~\ref{sec:plato} introduces the proposed Plato database system, which brings the real world model concept into SQL databases, therefore combining the healthy productivity aspects of SQL systems with the benefits of signal processing techniques.
=======
\paragraph{The healthy aspects of SQL databases in analytics}
%SQL systems have been a success in BI, OLAP...
Since the 80s SQL database systems gradually became the cornerstone of
the vast majority of Business Intelligence (BI) and On Line Analytical
Processing (OLAP) applications. 
Their success is primarily attributed to {\em declarative querying
(via SQL)}, {\em automatic query optimization} and the {\em
physical-logical independence}. In a typical database-driven
application architecture (see Figure~\ref{fig:db-driven-arch}), the
business intelligence application issues an SQL query over tables. The
tables are {\em logical} in the sense that they are just mathematical
relations, while the underlying storage, clustering and indexing
details constitute a {\em physical} layer, whose knowledge is not
needed for writing queries. The queries are declarative in the sense
that they only describe the desired result without describing the
algorithm that computes the result. A query optimizer automatically
decides which is the best plan for executing the query, taking into
consideration the physical organization of the storage. A database
administrator may alter the data structures and indices from time to
time in order boost the performance of the observed workload. However,
logical/physical separation dictates that fine-tuning changes at the ``physical" level do not require rewriting of the logical layer. Rather, the query processing speed will automatically benefit.
>>>>>>> 50bb71c4dc192ebf049064635822625d3bb6b25b


\subsection{The healthy aspects of SQL databases in analytics}
\label{sec:healthy-aspects}
In a typical database-driven application architecture (see Figure~\ref{fig:db-driven-arch}), the business intelligence application issues an SQL query over tables. As far as the application is concerned, the tables are {\em logical} in the sense that they are just mathematical relations. In contrast, the underlying data structures where the tables are stored, as well as the indices on the tables, constitute the {\em physical} layer, whose knowledge is not needed for writing queries. The queries are declarative in the sense that they only describe the desired result without describing the algorithm that computes the result. A query optimizer automatically finds a good plan (in effect, algorithm) for executing the query, taking into consideration the specifics of the physical layer. Declarativeness and automatic optimization increase {\em developer productivity} and {\em lower the sophistication bar} needed for writing analytics. Furthermore, they are a key enabler behind visual OLAP systems, where the (non-programming) user can easily aggregate, correlate, drill-down in the data, pivot, etc. Such systems (e.g., Cognos, Microstrategy) faced the relatively easy task of translating clicks into declarative queries, while the problem of optimizing these queries was left to the database.

\reminder{to Yannis P: The full truth is that also provided solutions to the problem of adding materialized views, which is actually connected to the problem of materializing models.
Further levels of abstraction are achieved with {\em views}, which are tables that capture the result of filtering, combining and aggregating the data of the logical tables in a way that simplifies the queries issued by the applications.
}

<<<<<<< HEAD
A database administrator may ``tune" the physical layer from time to time in order to boost the performance of the observed workload. However, logical/physical separation says that such tuning changes at the physical layer do not require rewriting of the logical layer. Rather, the query processing speed automatically benefits. Notice how logical/physical separation divides the expertise and tasks needed for the maintenance of the overall system, therefore making its maintenance and evolution very easy and inexpensive.
=======
\paragraph{The failures of SQL databases in spatiotemporal sensor data analytics}
%SQL systems fail in sensor data analytics
SQL systems fail in the space of predictive analytics involving
sensors that generate measurements in continuous space and time
%
\footnote{Indeed, one may argue that some of the discussed failures and respective remedies proposed by Plato, also apply to predictive analytics of non-sensor, non-spatiotemporal data. Nevertheless, the focus of this project is exclusively on spatiotemporal sensor data.
} 
\yoav{In this context it is important to distinguish between the 
{\em real world model} and the {\em sensor model}. The real world
model is a set of functions which map a {\em continuous coordinate
system} (such as time and space) to quantities such as temperature or
air quality. The sensor model defines the location of the sensors in
the real world model, the times in which they sample, and the value
sequences collected from the sensors.
The tension comes from the fact that the only source of data: the
measurements, are noisy and discrete in time and space, while the real
world model is defined for all coordinate values.}

Next, we argue that at the core of the failures of SQL databases is the fact that they lack a critical abstraction, which is the {\em Real World Model}.%
\footnote{The real world model is frequently called the ``physical model" as it corresponds to the physics of the real world. This proposal uses the term ``real world model" (instead of the term ``physical model") to avoid the confusion caused by the two different meanings of the word ``physical" in the sensors' signal processing and the database community.
}
Then we argue next that this omission leads to multiple failures: Tedious programming and system integration, lack of abstractions, which requires developers/analysts to jungle the high level, where they think of statistics, such as the correlation of two sequences, with the low level programming where they deal with the specifics of the model. The second failure is that often the massive amounts of measurements are used directly in the statistics, leading to a performance waste.

In general, the data contained in an SQL database have a one-to-one mapping to the captured real world objects and events. For example, a database may represent persons, which have an address, a name and a telephone number. Each represented person is supposed to be a corresponding unique entity in the world. Vice versa, as far as the database and the applications running on it are concerned, the represented entities, attributes and relationships constitute the entire real world of interest. On the other hand, the sensor data are mere samples (measurements) of the reality. The full reality of interest to the applications is captured by models that predict the quantity of interest at each point in a range of time and possibly also in an area of space. In practice, the gap between the measurements and the real world models is best crossed by learning algorithms that discover the parameters of the real world model. 

\eat{
meaningful events in a video recoding from a webcam
positioned over a highway are {\em not} the pixel values. The
meaningful events might be the number of vehicles in view, the speed
of vehicles in each lane or license plate numbers of the vehicle.
The entities in the real world do no relate to any individual pixel or
even group of pixels. They relate to {\em patterns} that appear across
different pixels at different times. It is by finding these patterns
that we can extract useful information from the video.
}
>>>>>>> 50bb71c4dc192ebf049064635822625d3bb6b25b

% the decoupling has led to nicely separated tools. We will see that such a clean architectural separation has collapsed.
% parallel
% logical-physical separation is pushed even further by auto-administrated systems \cite{}.


\begin{figure}
{\bf the conventional database-driven analytics architecture illustrating declarative query and optimizer that automatically finds how to utilize the data structures best.}
\caption{Database-driven analytics architecture}
\label{fig:db-driven-arch}
\end{figure}

\subsection{The failure of SQL databases in spatiotemporal sensor data analytics}
\label{sec:the-failure}
In conventional SQL databases, the data contained in the database have a one-to-one mapping to the captured real world objects and events. For example, a database may represent persons, which have an address, a name and a telephone number. Each represented person is supposed to be a corresponding unique entity in the world. Vice versa, as far as the database and the applications running on it are concerned, the represented entities, attributes and relationships constitute the entire real world of interest. In contrast to a conventional database, the sensor data are mere {\em measurements} (samples) of the reality. The full reality of interest to the analytics application is captured by models that predict the quantity of interest at each point in a range of time and/or space. In practice, the gap between the measurements and the real world models is best crossed by learning algorithms that discover the parameters of the real world model and are tuned to the type of the model. For example, when one models a temporal signal exhibiting periodicity, she will use the Fourier transform to discover the frequencies. Given the frequencies, the model can predict a value for any time.

In the absence of real world models in database systems, sensor analytics applications either do not use databases or they use them simply as a ``dumb" storage of measurements.
Given an analytics request, one has to first issue queries that retrieve the relevant measurements. Then she deploys signal processing learning algorithms on them to compute the model's parameters. Finally,  the (typically statistical) analysis is often performed over the model and its results are moved into the database. 
The healthy aspects of SQL database systems discussed in Section~\ref{sec:healthy-aspects} are no longer observed. Instead, the above process is characterized by low productivity as data move  in an out of the database. Furthermore, unlike the ecosystem of conventional SQL systems, the analysts of sensor data must be experts of many areas: signal processing, statistics, databases and often big data management and processing. As sensor data proliferate, the short supply of such multiarea experts and the low productivity in developing and maintaining sensor analytics will lead to a world where a lot of sensor data will be collected but only limited analysis will be performed at them, at the cost of missing the insights that these data can offer.

% One stupid approach: Work directly with the measurements

In recent years databases have been extended (typically via UDFs) with machine learning procedures that can allow one to execute such procedures without moving data in-and-out of the database. While such extensions are useful, we argue next that they miss the optimization opportunities that become available once models appear in the database as first class citizens and the query language and query processing are adjusted accordingly.

The net effect is that analytics on spatiotemporal sensor data currently require the involvement of developers who are essentially skilled in all of the (a) data filtering and combining, (b) modeling, and (c) low level data representation and related algorithms. 
% the decoupling shown earlier has collapsed.

\subsection{Plato: Models as a first class citizen of the database and query language}
\label{sec:plato}
The proposed Plato database has models as a ``first class citizen" abstract type of the database. At a first cut, a Plato model is a continous function over time and/or space that predicts a quantity of interest (eg, temperature, velocity, acceleration, air pollutants density etc) at any coordinate in a certain time/space area. Alternately, a database-minded person may perceive a Plato model as a table whose attributes are time and/or space coordinates and one or more types of quantities. The table has infinitely many tuples and the time/space coordinates form a key, i.e.,  given values for the time/space coordinates, each quantity has a unique associated value. 

% Queries that one can write.
% Challenge: develop the query languages

Consequently, Plato expands the applicability of the successful architecture of Figure~\ref{fig:db-driven-arch} into spatiotemporal sensor data management and analytics. The application issues declarative ``Plato SQL" queries, which utilize the models without being concerned about the specifics of how the models are actually represented in the storage nor the intricacies of how to most efficiently execute the queries. The following sections provide multiple examples of Plato SQL queries. In one example, a query predicts quantities at specific coordinates. In another example, models are correlated. In a third example, queries drill into specific time segments of the models; e,g., the afternoon times. In yet another example, models capturing the daily paths of individual persons are composed with models that provide the distribution of atmospheric polutants in San Diego County and the result is models of the polutant levels that these persons breathed; the query that composes the models is essentially just a simple SQL join. These model-related queries are seamlessly combined with the rest of the database, therefore making it easy to draw results about various segments of the entities represented in the database. For example, a single query can test an hypothesis regarding the inhalation of an atmospheric polutant by individuals and the asthma attack incidents (as recorded in the conventional medical record) of those indviduals.

\reminder{the next is rough}

\paragraph{Noise-reducted, additive model representations towards data quality}
Obviously, the {\em storage-level representation} (from now on called simply representation) of the models cannot be the infinetly many possible coordinates and their associated quantities. Rather, vast prior work in signal processing research teaches 

The effectiveness of such representations is connected to the physics of the signals. For example, the frequency domain representations generated by Fourier transforms capture the periodicity of certain signals. In another example, ARMA models capture the differential equations that govern the connection between neighbor points in time and space.

Plato's query language utilizes models in many ways: 

% models with Physics

The larger benefit of a Plato model (and the also the reason for naming the project Plato, in an allusion to Plato's Cave allegory%
\footnote{In Plato's Cave, a few prisoners are bound, since they were born, to look at a wall where shadows of the real world's objects appear. The prisoners perceive only the 2-dimensional world of the shadows and so they miss the deeper insights that a comprehension of the full 3-dimensional reality would offer them. Similarly, the sensor measurements that appear in a database are the mere projection of the real world. The insight in the world is captured by physical models (i.e., models that capture the world's physics). 
}
)

\paragraph{Noise-reducted models towards query processing efficiency}




\reminder{REMINDERS ON POINTS FOR INTRODUCTION}

\paragraph{Problems of current approaches to combining databases with statistical signal processing}
\begin{itemize}
\item (Done) Statistics and predictive analytics involve a tedious process where data are moved out of databases into statistical processing software (eg R) and then stored back.
%
\item None of the cornerstones of the success of SQL systems is available: No logical/physical separation, no declarative queries and no automatic optimization of storage and query processing.
%
\item (Done) Spatiotemporal sensor data are qualitatively different in a way that prevents the direct use of SQL, which has been tuned for discrete precise data: Unlike ERP and CRM business use cases, where the database data constitute the world, sensor data are mere measurements of the real world - potentially imprecise. Models (continuous functions over space and time) are the actual representation of the world and predict values at any coordinate. Query languages and database modeling has to facilitate the use of predicted values.
%
\item Sensor data are unnecessarilly big in their raw form. Yet signal processing research has known for long that data can be split into signal and the often unimportant noise, leading to high effective lossless compression. 
%
\end{itemize}

\paragraph{The benefits of a model-aware database}

\begin{itemize}
%
\item Plato brings models in the database. Models are continous functions that predict the quantity of interest at any coordinate. A model introduces knowledge of the external physical reality.
%
\item Well-developed theory and set of models and model generators that effectively capture many real world cases.
%
\item Holds promise to solve the problems above. Plato enables declarative queries and automatic optimization of storage and query processing.
%
\end{itemize}

\paragraph{Research Challenges}
\begin{itemize}
%
\item Specify declarative query languages that process models.
%
\item The database algorithms can work efficiently directly on the models. Specify a logically/physically-separated architecture where an optimizer is aware of the specifics of the model representation and chooses query processing algorithms accordingly.  For example, consider two models represented by their Fourier transform and a query that asks for their correlation. It is most efficient to compute directly on the frequency domain rather than bringing back to time domain.
%
\item Signal processing community provides a wide variety of model generators, with various guarantees on the loss information. Plato must semiautomate the choice of the appropriate model generator. This requires a classification of model generators with respect to their loss of information properties. More importantly, different queries may pose different needs of accuracy and of what constitues information loss. The semiautomation must take into account query workload. The query answering must utilize the best model for the problem. Granularity of queries can also play a role: Can you avoid computing the entire model and instead compute on the fly only the parts of the model that are of interest to the query? Which models are best fo such?
%
\item How do you incrementally maintain the model as the data change? Tradeoff between speed of convergence and computational resources used. Again, query workload must specify.
\end{itemize}

\paragraph{Use cases}

\subsection{Prior work on model-aware databases}
\reminder{to Yannis K: Please check and add to the representations I make about the prior work.}

\paragraph{Works from ML and statistical signal processing communities} \reminder{for Yoav}

\paragraph{MauveDB} Argued for the value of direct SQL processing. MauveDB uses models to predict values on a developer-specified grid. 

\begin{itemize}
\item Plato argues that models need no be discretized in the coordinates' grid. A fully virtual approach, where the model is perceived as a continous function, is easier for the developer and more opportune for the optimizer. For example, consider two models represented by their Fourier transform and a query that asks for their correlation. It is most efficient to compute directly on the frequency domain rather than bringing back to time domain.
%
\item Did not investigate the connections between (a) choosing models and query workload and (b) query guarantees given chosen models.
\end{itemize}


\paragraph{Function Db (MIT)} 
\begin{itemize}
\item Showed that in the case of models captured by polynomials better compression and faster processing is achieved by working directly on polynomials.
\end{itemize}


\paragraph{Aberer}
\reminder{for Yannis K: It seems he promised a lot and did a little. What do we write about this?}


\paragraph{Approximate query processing}
