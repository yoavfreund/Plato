\section{Relevant NSF Prior Work of the PIs}
\label{sec:prior}
PI Papakonstantinou's expertise is in data-driven middleware and data integration software. A theme of his research works has been the delivery of declarative, query-based platforms for solving problems in data integration and/or application development. The small NSF IIS awards {\bf 1018961} and {\bf 1219263} (PI'd by Papakonstantinou, publications [1], [3] of Papakonstantinou bio) develop a declarative, SQL-based  programming system, named FORWARD, which delivers database-driven Ajax pages as rendered SQL views, obliterating the need for low level, low productivity Ajax visualizations coding. FORWARD has already been deployed in industrial use cases%
\footnote{
A limited software release of the FORWARD framework has been used in local pharmaceutical companies (Ferring, Pfizer) since 2012 and the current version is now in pilot testing at Teradata, for the purpose of providing a high productivity visualization framework.
}
and makes the case for the productivity and simplicity of declarative programming, which is a theme pursued in the current project also, albeit combining different technologies with SQL.  
PI Papakonstantinou is a co-PI of NSF SHB award {\bf 1237174} (project DELPHI), which has provided a key inspiration towards Plato, as explained in Section~\ref{sec:use-cases} and as also articulated in the publication [4] of Papakonstantinou bio.

Co-PI Freund is an expert in the field of machine learning, statistics
and information theory, his best known work is the Adaboost learning
algorithm, which he co-invented with Dr Robert Schapire. In recent
years his main focus has been on the analysis of large collections of
sensory data, image collections and data collected using microphone
arrays.

Co-PI Freund is an expert on online learning algorithms, these are
algorithms for learning from a stream of data which make no assumption
about the way the data is generated.

Co-PI Freund is is collaborating with researchers from the San Diego
Super Computer Center and with Research firm OSI-Soft Inc. on a large
scale system for the analysis of sensor measurements collected from
buildings in UCSD over the last two years. This data consists of
80,000 sensor readings and amounts to about 500 TB of information.

Co-PI's is the PI on NSF grant number 1162581 {\bf RI: Medium: Quantifying
and utilizing confidence in machine learning}. The focus of this work
is on identifying measures of confidence that hold with the little or
no prior assumptions. Such confidence prediction measures are
important for active learning and for the work planned to be done on
this proposal. 
