\subsection{Use cases}
\label{sec:use-cases}
This project (and many of this proposal's examples) have been inspired by interactions of the PIs with the sensor data sets of two prior UCSD projects: the Energy Dashboard project ({\tt http://energy.ucsd.edu}) and the DELPHI project ({\tt http://delphi.ucsd.edu}). 

The UCSD San Diego Energy Dashboard project (PI'd by Professor Rajesh Gupta) has planted 80,000 ``smart building" sensors in UCSD and collects a vast volume of data. The PIs of the Plato proposal obtained the data during the recent months%
\footnote{co-PI Freund collaborates with researchers from the San Diego
Super Computer Center and with Research firm OSI-Soft Inc. on a large
scale system for the analysis of the 500 TB of sensor measurements collected from
buildings in UCSD over the last two years.
} 
and ran preliminary experiments that validate the appropriateness of these data for the purposes of Plato. In particular, the PIs' team ran SVD and wavelet transformations on the acquired data and validated that the resulting highly compressed, reduced-noise model representations were sufficient for efficiently answering queries on expected values and correlations. While the limited extent of the experimentation does not yet constitute a proof-of-concept of the validity of Plato's approach, this data set is clearly appropriate for Plato's experimentation.

The NSF-funded DELPHI project (Data E-platform Leveraged for Patient Empowerment and Population Health Improvement, PI'd by Professor Kevin Patrick) collects environmental sensor data. Some sensors belong to the San Diego County and measure the concentration of multiple air polutants. An atmosperic polutants model has been obtained from the San Diego Supercomputing Center: Given the measurements, it predicts the concentration of a pollutant at given space coordinates and time. Other sensors belong to the individuals and track %
\footnote{The PI of the Plato project is one of the five co-PIs of the DELPHI project} 
the places where an individual walks or runs every day. The project will soon also obtain asthma attack-related data from patients who have the respective inhaling activity sensor. The ``hardwired" approach by which these data are going to be correlated with each other (i.e., how do asthma attacks correlate with the air an individual breathes), analyzed and visualized reduces the productivity of population level studies. Since Plato cannot be ready by the time that these population studies/analytics will need to be contacted, the first round of DELPHI analytics will be built using the current state of the art in sensor analytics. The Plato project's goal is to later replicate the same analytics using Plato and compare both results' quality and analyst/developer productivity.